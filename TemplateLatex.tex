\documentclass[12pt]{extarticle}
\usepackage[utf8]{vietnam}
\usepackage[colorlinks=true, allcolors=black]{hyperref}
\usepackage{fancyhdr} % Gói tạo header/footer
\usepackage{graphicx} % Gói chèn hình ảnh
\usepackage[a4paper, top=2cm, bottom=2.5cm, left=2cm, right=2cm]{
  geometry
} % Tùy chỉnh lề
\pagestyle{fancy}% Kích hoạt fancyhdr
\fancyhead{}
\fancyfoot[c]{\thepage}
\renewcommand{\headrulewidth}{0pt} %Loại bỏ gạch chân của header
\setlength{\parindent}{1cm} %thụt lề
\setlength{\parskip}{0cm}
\usepackage{indentfirst}
\usepackage{amsmath} %Toán
\usepackage{amssymb}
\usepackage{amsthm}
\usepackage{siunitx}
\makeatother
\renewcommand{\familydefault}{\sfdefault} %Đặt font
\theoremstyle{definition}
\newtheorem{cau}{Câu}
\usepackage{enumitem}
\newenvironment{muc}{\begin{enumerate}[label=\alph{*})]}{\end{enumerate}}
\usepackage{enumitem} %không thụt lề
\newenvironment{tracnghiema}{\begin{enumerate}[label=\textbf{\Alph{*}.}]}{\end{enumerate}}
\newenvironment{tracnghiem}[1]{\begin{multicols}{#1}\begin{enumerate}[label=\textbf{\Alph{*}.}]}{\end{enumerate}\end{multicols}}
\usepackage{multicol}
\usepackage{titling}
\usepackage{titlesec} % Gói để chỉnh section
\usepackage{lipsum} % Gói để tạo văn bản giả lập
\renewcommand{\maketitle}{
\vspace*{1cm}
\begin{center} % Căn giữa
  {\Huge \MakeUppercase{\textbf{\thetitle}}} % Tiêu đề rất lớn, in đậm
\end{center}
\vspace{0.5cm}
}
% Căn phải tiêu đề section và thêm gạch dưới
\renewcommand{\thesection}{\Roman{section}}
\renewcommand{\thesubsection}{\arabic{subsection}. }
\renewcommand{\thesubsubsection}{\arabic{subsection}.\arabic{subsubsection}}
\newcommand{\sectionunderline}{%
\vspace{-0.75cm} % Khoảng cách trước gạch ngang
\noindent
\rule{\textwidth}{0.4pt} % Gạch ngang dưới tiêu đề
\vspace{0.5cm} % Khoảng cách sau gạch ngang
}
\titleformat{\section}
{\LARGE\bfseries\raggedleft} % Font lớn, đậm, căn phải
{}
{0 cm} % Không có khoảng cách giữa số và tiêu đề
{\MakeUppercase} % Chuyển tiêu đề thành in hoa
[\sectionunderline] % Gọi lệnh gạch ngang sau tiêu đề

\titleformat{\subsection}
{\centering\Large\bfseries} % Căn giữa, chữ lớn, đậm
{\thesubsection} % Hiển thị số
{00cm} % Khoảng cách giữa số và tiêu đề
{\MakeUppercase} % Chuyển tiêu đề thành chữ IN HOA
\titlespacing{\subsection}{0pt}{0.5cm}{-0cm}
\titleformat{\subsubsection}
{\large\bfseries} % Chữ lớn, in đậm
{\thesubsubsection} % Hiển thị số (1.1.1, 1.1.2, ...)
{0.5cm} % Khoảng cách giữa số và tiêu đề
{} % Không thay đổi nội dung (giữ nguyên chữ thường)
% Tùy chỉnh khoảng cách trước và sau subsubsection
\titlespacing{\subsubsection}{0pt}{0.25cm}{-0cm}
\theoremstyle{definition}
\newtheorem{dinhnghia}{Định nghĩa}
\usepackage{environ}
\NewEnviron{note}{%
  \begin{flushright}
    \colorbox{gray!15}{%
      \begin{minipage}{\dimexpr\linewidth - 3cm\relax}
        %\color{gray}
        \setlength{\parindent}{2em}
        \setlength{\parskip}{0.5cm}
        \BODY % nội dung người dùng viết trong môi trường
      \end{minipage}
    }
  \end{flushright}
}
\everymath{\displaystyle}
\usepackage{pgfplots}
\usepackage{mathrsfs}
\usetikzlibrary{arrows}
\usetikzlibrary{angles,quotes}
\DeclareMathOperator{\lcm}{lcm}
