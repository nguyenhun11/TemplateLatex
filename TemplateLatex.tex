%===CHUNG
\documentclass[12pt]{extarticle}
\usepackage[utf8]{vietnam}
\usepackage[colorlinks=true, allcolors=black]{hyperref}
\renewcommand{\familydefault}{\sfdefault} %Đặt font
\usepackage{graphicx} % Gói chèn hình ảnh

%===LỀ, ĐOẠN
\usepackage[a4paper, top=2cm, bottom=2.5cm, left=2cm, right=2cm]{geometry} % Tùy chỉnh lề
\setlength{\parskip}{0cm}
\setlength{\parindent}{1cm} %thụt lề
\usepackage{indentfirst}

%===HEADER, FOOTER
\usepackage{fancyhdr} % Gói tạo header/footer
\pagestyle{fancy}% Kích hoạt fancyhdr
\fancyhead{}
\fancyfoot[c]{\thepage}
\renewcommand{\headrulewidth}{0pt} %Loại bỏ gạch chân của header

%===TOÁN
\usepackage{amsmath} %Toán
\usepackage{amssymb}
\usepackage{amsthm}
\usepackage{siunitx}
\everymath{\displaystyle} %Luôn ở dạng displaystyle
\usepackage{tikz}
\usepackage{pgfplots}
\usepackage{mathrsfs}
\usetikzlibrary{arrows}
\usetikzlibrary{angles,quotes}

%===MÔI TRƯỜNG MỚI====
\usepackage{environ} % Gói định nghĩa lại môi trường
\usepackage{multicol} %Gói chia cột
\usepackage{enumitem} %Định nghĩa môi trường mục
\usepackage{changepage} %chỉnh lề
\theoremstyle{definition} %Tạo môi trường mới

%---Định nghĩa
\newtheorem{dinhnghia}{Định nghĩa}

%---Câu hỏi
\newtheorem{cau}{Câu} %Môi trường câu hỏi

%---Mục a), b)
\newenvironment{muc} %Môi trường mục a), b)
    {\begin{enumerate}[label=\alph*)]} 
    {\end{enumerate}} 

%---Mục 1.1, 1.2
\newlist{MUC}{enumerate}{1}
\setlist[MUC,1]{label=\thecau.\arabic*, ref=\thecau.\arabic*.}

%---Trắc nghiệm
\NewEnviron{tracnghiem}[1]{%
    \ifthenelse{\equal{#1}{1}}{%
        \begin{enumerate}[label=\textbf{\Alph*.}]
            \BODY
        \end{enumerate}
    }{%
        \begin{multicols}{#1}
            \begin{enumerate}[label=\textbf{\Alph*.}]
                \BODY
        \end{enumerate}
        \end{multicols}
    }%
}
%---Trắc nghiệm a (các tài liệu cữ)
\newenvironment{tracnghiema}{\begin{enumerate}[label=\textbf{\Alph{*}.}]}{\end{enumerate}}

%---Ghi chú
\NewEnviron{note}{%
    \begin{flushright}
        \colorbox{gray!15}{% màu xám
        \begin{minipage}{\dimexpr\linewidth -3cm\relax} %Chỉnh độ thụt của khung
            %\color{gray} % Màu chữ, bỏ comment nếu muốn chữ màu xám
            \setlength{\parindent}{0cm} %Thụt lề đầu dòng
            \setlength{\parskip}{0cm} %Cách đoạn
            \BODY % nội dung người dùng viết trong môi trường
        \end{minipage}
        }
    \end{flushright}
}

%===THIẾT LẬP LẠI CÁC PHẦN
\usepackage{titling}
\usepackage{titlesec} % Gói để chỉnh section
\usepackage{lipsum} % Gói để tạo văn bản giả lập

%---Tiêu đề
\renewcommand{\maketitle}{
    \vspace*{1cm}
    \begin{center} % Căn giữa
        {\Huge \MakeUppercase{\textbf{\thetitle}}} % Tiêu đề rất lớn, in đậm
    \end{center}
    \vspace{0.5cm}
}

%---Section
\renewcommand{\thesection}{\Roman{section}}
\newcommand{\sectionunderline}{%
    \vspace{-0.75cm} % Khoảng cách trước gạch ngang
    \noindent
    \rule{\textwidth}{0.4pt} % Gạch ngang dưới tiêu đề
    \vspace{0.5cm} % Khoảng cách sau gạch ngang
}

\titleformat{\section}
    {\LARGE\bfseries\raggedleft} % Font lớn, đậm, căn phải
    {}
    {0 cm} % Không có khoảng cách giữa số và tiêu đề
    {\MakeUppercase} % Chuyển tiêu đề thành in hoa
    [\sectionunderline] % Gọi lệnh gạch ngang sau tiêu đề

%---Subsection
\renewcommand{\thesubsection}{\arabic{subsection}. }
\titleformat{\subsection}
    {\centering\Large\bfseries} % Căn giữa, chữ lớn, đậm
    {\thesubsection} % Hiển thị số
    {00cm} % Khoảng cách giữa số và tiêu đề
    {\MakeUppercase} % Chuyển tiêu đề thành chữ IN HOA
    
\titlespacing{\subsection}{0pt}{0.5cm}{-0cm}

%---Subsubsection
\renewcommand{\thesubsubsection}{\arabic{subsection}.\arabic{subsubsection}}
\titleformat{\subsubsection}
    {\large\bfseries} % Chữ lớn, in đậm
    {\thesubsubsection} % Hiển thị số (1.1.1, 1.1.2, ...)
    {0.5cm} % Khoảng cách giữa số và tiêu đề
    {} % Không thay đổi nội dung (giữ nguyên chữ thường)
\titlespacing{\subsubsection}{0pt}{0.25cm}{-0cm}


%===CÁC LỆNH MỚI====

%---Kẻ n dòng chấm \dotsline{n}
\newcommand{\dotsline}[1]{%
    \par
    \vspace{1em}
    \noindent
    \foreach \i in {1,...,#1}{%
        \mbox{}\dotfill\\[1em]%
    }%
}
